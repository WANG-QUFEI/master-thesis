% CREATED BY DAVID FRISK, 2016
\oneLineTitle\\
% \oneLineSubtitle\\
QUFEI WANG\\
Department of Computer Science and Engineering\\
Chalmers University of Technology and University of Gothenburg\setlength{\parskip}{0.5cm}

\thispagestyle{plain}			% Supress header 
\setlength{\parskip}{0pt plus 1.0pt}
\section*{Abstract}
We present in this paper a simple dependently typed language. The basic form of this language contains only a universe of small types, variables, lambda abstraction, function application and dependent product as its syntax. There is no data types and mutual recursive/inductive definitions is not supported. This language could be viewed as a lambda calculus extended with dependent types and constant definitions. The focus of this project is not on the expressiveness of the language but on the study of a locking/unlocking mechanism where the definitions of constants could be handled efficiently during the type checking process. Keeping the syntax simple helps us focus on our purpose and make the implementation elegant at the same time. We later enriched the language with a module system not to increase its expressiveness, but to observe how the locking/unlocking mechanism should be adjusted for the introduction of namespaces to the variables. The outcome of our work is a REPL(read-evaluate-print-loop) program through which a source file of our language could be loaded and type checked. The program also provides auxiliary functions for the user to experiment with and observe the effect of the locking/unlocking mechanism. The syntax of our language is specified by the BNF converter and the program is implemented in Haskell. We hold the expectation that our work could contribute to the development of the proof systems that are based on the dependent type theory.

% KEYWORDS (MAXIMUM 10 WORDS)
\vfill
Keywords: computer science, dependent type theory, functional programming, type checker.

\newpage				% Create empty back of side
\thispagestyle{empty}
\mbox{}