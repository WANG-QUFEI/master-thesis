\chapter{Terminology}
In order to make clear of the potential ambiguity or unnecessary confusion over the words we choose to use in this paper, we list below the terminology we use and their meanings:
\begin{itemize}
  \item \textbf{Declaration:} A \emph{declaration} has either the form $x : A$ or $x : A = B$. The latter is also referred, rather frequently, as a \emph{definition}. Sometimes when we want to make a distinction between these two forms, we also use the word `declaration' specifically to indicate a term of the former form. 
  \item \textbf{Definition:} A \emph{definition} is a term of the form $x : A = B$, meaning that $x$ is an element of type $A$, defined as $B$. Sometimes when we want to talk about the components of a specific definition, we also use the word `definition' specifically to indicate the part of $B$, e.g., ``the definition of $x$ is $B$'' .  
  \item \textbf{Constant:} A \emph{constant} is the entity that a name or identifier used in a declaration refers to, e.g., the entity that $x$ in $x : A$ or $x : A = B$ refers to.
  \item \textbf{Variable:} A synonym for \emph{constant}. More often, the word `variable' is used to refer to the variable bound in a $\lambda$-abstraction, like the variable $x$ in $\lambda x . A$. In most cases, these two words are interchangeable. 
\end{itemize}
